\documentclass{article}

% various packages and settings {{{

% math packages
\usepackage{amsmath, amsfonts, mathtools}

% other important packages (should stay on top)
\usepackage{graphicx, multicol, xcolor}

% change to German
\usepackage[german]{babel}

% utf8 characters
\usepackage[utf8]{inputenc}

% font
\usepackage[scaled]{helvet}
\renewcommand{\familydefault}{\sfdefault}
\usepackage[]{fontspec}

% fontsize
\usepackage[12pt]{extsizes}

% no paragraph indents
\setlength{\parindent}{0pt}

% better hyphenation
\usepackage[final]{microtype}
\usepackage{csquotes}

% import line spacing
\usepackage{setspace}

% page format
\usepackage{geometry}
\geometry{
    a4paper,
    left=25mm,
    right=25mm,
    top=25mm,
    bottom=25mm
}
    
% page header
\usepackage{fancyhdr}
\pagestyle{fancy}
\fancyhf{}
\lhead{\textbf{\today}}
\rhead{\textbf{Schwingungen}}

% colorful boxes
\usepackage{mdframed}

\setcounter{secnumdepth}{0} % no section numbering

% }}}

% custom commands {{{
\usepackage{environ}

% define custom formula environment
\NewEnviron{formulas}{
    \vspace{-2.5em}
    \Large
    \begin{align*}
        \BODY
    \end{align*}
    \normalsize
}

% same formula env, but with red border
\NewEnviron{Formulas}{
    \begin{mdframed}[linecolor=red]
        \vspace{-2.5em}
        \Large
        \begin{align*}
            \BODY
        \end{align*}
        \normalsize
    \end{mdframed}
}

% normal-sized text inside the formulas
\newcommand{\normaltext}[1]{
    \normalsize\text{#1}\Large
}

% }}}

\begin{document}

% set line spacing
\setstretch{1.5}

\section{Versuche zu Feder- und Fadenpendel}

Um Aufschluss über die Zusammenhänge von Masse ($m$), Federstärke /
Federkonstante ($D$),
Startamplitude ($s_0$) beim Federpendel und Masse ($m$), Fadenlänge ($l$),
Startamplitude ($s_0$) beim Fadenpendel in Bezug auf die zu Erwartende
Schwingungsdauer zu erhalten lassen sich 2 Experimente durchführen.

Faden und Feder lassen sich an einem Stativ befestigen. Anschließend befestigt man ein
Massestück an Faden und Feder, sodass man nun 2 funktionierende Pendel hat.
Zur Messung der Schwingungsdauer (T) läßt sich nun eine Lichtschranke verwenden.
Beim Federpendel bietet es sich dabei an, etwas vom Massestück herausragendes zu nutzen
wie z. B. ein Stück Papier, damit nicht die Feder detektiert wird von der Lichtschranke.

In verschiedenen Experimentteilen lassen sich nun die Größen Masse ($m$),
Federstärke / Federkonstante ($D$), Fadenlänge ($l$) und Startamplitude ($s_0$) variieren.
So lassen sich anschließend Messwerte erkennen. Besonders einfach ist dies, wenn man
eine Lichtschranke mit guten digitalen möglichkeiten nutzt, so dass die Daten von einem
digitalen Gerät (e. g. Tablet) direkt z. B. als Graph dargestellt werden.

\clearpage

\section{Federpendel}

\subsection{Herleitung der Formel zur Berechnung der Schwingungsdauer (T) beim Federpendel}

Diesem Ansatz liegt der zugrunde, dass beim Federpendel das lineare Kraftgesetz gilt.

\begin{formulas}
    F_R = -D \cdot s_0 && F_R \propto s_0
\end{formulas}

Die Kraft läßt sich auch durch $F = m \cdot a$ beschrieben.
Die Beschleunigung ist die 2. Ableitungen der Strecke $a = \ddot{s}$. Dies lässt sich
einsetzen.

\begin{formulas}
    m \cdot \ddot{s} = -D \cdot s_0
\end{formulas}

Da die Strecke jeweils abhängig zur Zeit ist, lassen sich in diesem Fall $\ddot{s}$ und
$s_0$ als Sinusfunktion mit $t$ als Parameter darstellen.

\begin{formulas}
    m \cdot \ddot{s}(t) = -D \cdot s(t)
\end{formulas}

Vereinfacht kann man von einer ungedämpften Schwingung ausgehen, sodass wir eine
normale Sinusfunktion mit $s_0$ als Amplitude haben. Daraus lässt sich dann auch die
Ableitung bilden.

\begin{formulas}
    s(t) = s_0 \cdot \sin (\frac{2\pi}{T} \cdot t) \\
    \ddot{s}(t) = s_0 \cdot \sin (\frac{2\pi}{T} \cdot t) \cdot -(\frac{2\pi}{T})^2
\end{formulas}

Nun setzt man diese beiden Funktionen ein und formt um.

\begin{formulas}
    m \cdot -(\frac{2\pi}{T})^2 \cdot s_0 \cdot \sin (\frac{2\pi}{T} \cdot t) & =
        -D \cdot s_0 \cdot \sin (\frac{2\pi}{T} \cdot t)
                                                                              & | : s_0 \cdot \sin (\frac{2\pi}{T} \cdot t) \\
        m \cdot -(\frac{2\pi}{T})^2                                               & =
    -D                                                                        & | \cdot -1, : m                             \\
        (\frac{2\pi}{T})^2                                                        & =
    \frac{D}{m}                                                               & | \sqrt{\phantom{x}}                        \\
        \frac{2\pi}{T}                                                            & =
    \sqrt{\frac{D}{m}}                                                        & | \normaltext{Kehrbrüche}, : 2\pi     \\
    T                                                                         & =
        2\pi \cdot \sqrt{\frac{m}{D}}
\end{formulas}

\clearpage

\section{Fadenpendel}

$T = 2\pi \cdot \sqrt{\frac{l}{g}}$

\end{document}
