\section{Rekursive Problemlösung}

\subsection{Divide and Conquer Prinzip}

Beim ``Divide and Conquer'' (Teile und Herrsche) Prinzip wird ein größeres, schwereres Problem
in viele Kleine zerlegt, die dann jedes für sich einfacher gelöst werden können.
Dies geschieht häufig mit rekursiven Funktionen z. B. bei Such- und Sortierverfahren
(e.g. Binäre Suche, Quicksort).

\subsection{Fachbegriffe der Rekursion}

\subsubsection{Abbruchbedingung (Rekursionsanker)}

Die Abbruchbedingung verhindert die unendliche Ausführung der rekursiven Funktion.
Sie stellt dabei die Basis der rekursiven Definition dar und wird meist nach dem vollständigen
rekursiven Abstieg erreicht.

\begin{lstlisting}
sum(n)
    if n=0 # Rekursionsanker
        return 0
    else    # Rekursionsabstieg
        return n + sum(n-1)
\end{lstlisting}

\subsubsection{Rekursiver Abstieg/Aufstieg}
\label{sec:rekursion_ab_auf}

Der rekursive Abstieg bezeichnet das rekursive Selbstaufrufen der Funktion bis zum
erreichen des Rekursionsankers. Danach beginnt der Rekursionsaufstieg, bei dem
das Ergebnis berechnet bzw. zurückgegeben wird.

\begin{lstlisting}
sum(3) = 3 + sum(2)             # Beginn Rekursiver Abstieg
    sum(2) = 2 + sum(1)
        sum(1) = 1 + sum(0)
            sum(0) = 0          # Auslösen Rekursionsanker
        sum(1) = 1 + 0 = 1
    sum(2) = 2 + 1 = 3
sum(3) = 3 + 3 = 6              # Ende Rekursiver Aufstieg
\end{lstlisting}

\subsection{Rekursionsarten}

\subsubsection{Aufsteigende Rekursion (Einzelrekursion)}

Bei der Aufsteigenden Rekursion wird das Ergebnis der rekursiven Funktion
beim Rekursionsaufstieg berechnet. Dies ist beispielhaft bei \hyperref[sec:rekursion_ab_auf]{Rekursiver Abstieg/Aufstieg} beschrieben.

\clearpage

\begin{lstlisting}
sum(n)
    if n=0
        return 0
    else
        return n + sum(n-1)
\end{lstlisting}

\subsubsection{Endrekursion (Einzelrekursion)}

Bei der Endrekursion erfolgt die Berechnung beim Rekursionsabstieg und endet,
wenn die Abbruchbedingung erreicht ist. Beim Rekursionsaufstieg wird nur das
Ergebnis durchgereicht.

\begin{lstlisting}
sum(m, n)
    if n=0
        return m
    else
        return sum(m+n, n-1)
\end{lstlisting}

\subsubsection{Baumartige Rekursion (Mehrfachrekursion)}

Bei der Baumartigen bzw. Kaskadierenden Rekursion ruft sich eine Funktion pro Funktionsaufruf
mehrmals rekursiv auf. Der Rückgabewert setzt sich aus der Kombination 2er oder mehr Aufrufe
der Funktion zusammen.

\begin{lstlisting}
bintree_sum(t)
    if t.leaf()
        return t.value()
    else
        return bintree_sum(t.left()) + bintree_sum(t.right())
\end{lstlisting}

\subsubsection{Weitere Rekursionsarten (Nicht Kerncurriculum relevant)}

Indirekte Rekursion bezeichnet 2 oder mehr Funktionen, die sich zwar
nicht direkt selbst sondern gegenseitig aufrufen, wodurch auch Rekursion
zustande kommen kann.

\begin{lstlisting}
f(x)
    ...
    return g(x - 1)

g(x)
    ...
    return f(x - 1)
\end{lstlisting}

Geschachtelte Rekursion bezeichnet die Schachtelung von rekursiven Aufrufen
innerhalb der Funktion.

\begin{lstlisting}
f(x)
    ...
    return f(x - f(x))
\end{lstlisting}
