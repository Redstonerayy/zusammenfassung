

\section{Grammatiken}

\subsection{Formale Sprachen}

Formale Sprachen bestehen auf \textbf{Wörtern} die wiederum aus den Symbolen eines Alphabets
bestehen. Automaten können diese Wörter erkennen und Grammatiken können diese
Sprachen beschreiben (und somit festlegen).

Eine formale Sprache $L$ wird mit einer Grammatik $G$ beschrieben.
Sie ist über die Mengen an \textbf{Nicht-Terminal Symbolen} $N$ und
\textbf{Terminal Symbolen bzw. Alphabet} $T$ oder $\Sigma$ definiert. Es gibt ein \textbf{Startsymbol} $S$ und
\textbf{Produktionsregeln}, die den Aufbau der Wörter definieren.

\begin{figure}[h]
    \centering
    \Large
    \begin{align*}
        G_{Din} & = \{N, \Sigma, P, S\} \\
        N & = \{S, T, Z\} \\
        \Sigma & = \{A, B, C, D, 0, 1, 2, 3, 4, 5, 6, 7, 8, 9\} \\
        \\
        S & \rightarrow TZ \mkern7mu | \mkern7mu T10 \\
        T & \rightarrow A \mkern7mu | \mkern7mu B \mkern7mu | \mkern7mu C \mkern7mu | \mkern7mu D \\
        Z & \rightarrow 0 \mkern7mu | \mkern7mu 1 \mkern7mu | \mkern7mu 2 \mkern7mu | \mkern7mu 3 \mkern7mu | \mkern7mu 4 \mkern7mu | \mkern7mu 5 \mkern7mu | \mkern7mu 6 \mkern7mu | \mkern7mu 7 \mkern7mu | \mkern7mu 8 \mkern7mu | \mkern7mu 9
    \end{align*}
    \caption*{\textbf{Grammatik zum erkennen von DIN Formaten}}
\end{figure}

\normalsize
Beispielsweise beschreibt die Grammatik $G_{Din}$ die Bezeichnungen der DIN Formate.
Vom Startsymbol $S$ ausgehend lassen sich alle Formatebezeichnungen bilden.
Dabei ist wichtig, dass eine Grammatik alle Bezeichnungen bilden kann, aber nicht
noch andere Wörter gebildet werden können.

\clearpage

\subsection{Chomsky Hierarchie}

Die Hierarchie der Grammatiken nach Chomsky unterteilt Grammatiken in Typen von 0
bis 3, wobei es mit höherer Zahl mehr Einschränkungen für die Produktionsregeln gibt.
$\alpha, \beta, \gamma$ beschreiben in diesem Fall eine beliebige Ananeinanderreihung
von Terminalen und Nicht-Terminalen.
$A, B$ beschreiben Nicht-Terminal Symbole.
$a$ beschreibt in Terminal Symbol.

\textbf{Typ 0: Unbeschränkte Grammatik}
\Large
\begin{align*}
    \alpha \rightarrow \beta
\end{align*}
\normalsize

\textbf{Typ 1: Kontextsensitive Grammatik}
\Large
\begin{align*}
    \alpha A \beta \rightarrow \alpha \gamma \beta
\end{align*}
\normalsize

\textbf{Typ 2: Kontextfreie Grammatik}
\Large
\begin{align*}
    A \rightarrow \gamma
\end{align*}
\normalsize

\textbf{Typ 3: Reguläre Grammatik (rechtsregulär)}
\Large
\begin{align*}
    A \rightarrow aB \\
    A \rightarrow a \\
    A \rightarrow \varepsilon
\end{align*}
\normalsize

Bei der Regulären Grammatik kann diese auch linksregulär sein ($A \rightarrow Ba$),
allerdings dürfen die Produktionsregeln nicht gemischt sein.
Zusätzlich gibt es auch noch erweiterte Reguläre Grammatiken, die gleichwertig und
umwandelbar in Reguläre Grammatiken sind, wo jedoch mehr als 1 Terminalsymbol zugelassen
sind in der Beschreibung, was eine Grammatik potenziell verkürzt oder vereinfacht
($A \rightarrow aaa...B$).

\clearpage

\subsection{Umwandlung zwischen Automaten und Grammatiken}

Zwischen Automaten und Grammatiken lässt sich auf verschiedenen Wegen umwandeln,
auch weil es mehr als einen Automaten oder eine Grammatik für eine Formale Sprache
geben kann. Zusätzlich kommt es aber auch auf die Art von Grammatik und Automat an.
Für die Umwandlung von Deterministischen Endlichen Automaten zu Rechtregulären Grammatiken
lässt sich folgendes Schema anwenden.
Häufig lassen sich danach oder im Prozess noch Vereinfachungen oder Verkürzungen vornehmen. 

\begin{itemize}
    \item[1.] Die Zustände werden zur Menge der Nicht-Terminal Symbole
    \item[2.] Das Eingabealphabet $\Sigma$ der Automats wird zur Menge der Terminal Symbole
    \item[3.] Der Anfangszustand ist das Startsymbol
    \item[4.] Ein Zustandsübergang \large$z_1 \overset{a}{\rightarrow} z_2 \mkern7mu$\normalsize wird zur Produktionsregel \large$z_1 \rightarrow az_2\normalsize$\normalsize
    \item[5.] Für jeden Entzustand wird eine Produktionsregel mit $\varepsilon$ benötigt (\large$z \rightarrow \varepsilon$\normalsize) 
\end{itemize}
