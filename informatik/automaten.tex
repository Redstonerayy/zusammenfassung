\section{Automaten}

\subsection{Deterministische Endliche Automaten (DEA)}

Deterministische Endliche Automaten werden vorallem dazu genutzt um zu prüfen,
ob Strings einem bestimmten Schema entsprechen, also z. B. bestimmte Buchstabenfolgen
enthalten sind (vgl. Reguläre Ausdrücke).
Diese bestehen aus Zuständen $Q$ und Übergängen mit 1 Startzustand und einem oder
mehreren Endzuständen. Außerdem ist für jeden Automat ein Eingabealphabet $\sum$
definiert, was alle Symbole enthält, die in der Eingabe (dem String) vorkommen können.
Bei der graphischen Darstellung eines DEA wird allerdings meist nur das Eingabealphabet
$\sum$ zusätzlich mit angegeben, da sich der Rest aus der Darstellung ergibt.

\begin{tikzpicture}[shorten >=1pt,node distance=2cm,auto]
    \tikzstyle{every state}=[fill={rgb:black,1;white,10}]
  
    \node[state,initial]   (s)                      {$s$};
    \node[state,accepting] (q_1) [below left of=s]  {$q_1$};
    \node[state]           (q_2) [below of=q_1]     {$q_2$};
    \node[state,accepting] (r_1) [below right of=s] {$r_1$};
    \node[state]           (r_2) [below of=r_1]     {$r_2$};
  
    \path[->]
    (s)   edge              node {a} (q_1)
          edge              node {b} (r_1)
    (q_1) edge [loop left]  node {a} (   )
          edge [bend left]  node {b} (q_2)
    (q_2) edge [loop left]  node {b} (   )
          edge [bend left]  node {a} (q_1)
    (r_1) edge [loop right] node {b} (   )
          edge [bend left]  node {a} (r_2)
    (r_2) edge [loop right] node {a} (   )
          edge [bend left]  node {b} (r_1);
\end{tikzpicture}



image
explanation

\subsection{Deterministische Keller Automaten (DKA)}

image
explanation

\subsection{Mealy Automat}

image
exlanation
