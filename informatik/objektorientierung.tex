\section{Objektorientierung}
\subsection{Klassen}

Die Objektorientierte Programmierung mit Klassen hat zum Ziel,
reale Zusammenhänge zu modellieren und Dinge zu strukturen und
zu organisieren. Dazu gibt es Klassen mit Attributen und Methoden
mit denen sich Objekte (Instanzen der Klasse) erzeugen lassen.
Eine Klasse ist ein Schemata zur Erzeugung eines Objekts.

\pgfsetlayers{package0, main}

\vspace*{0.5cm}
\begin{figure}[H]
\begin{center}
\hspace*{-2cm}
\begin{minipage}{.7\textwidth}
    \begin{lstlisting}[language=python]
    class Car:
        def __init__(self, name: str, speed: float):
            self.name = name      # Attribut
            self.speed = speed
            self.__km_driven = 0
            
        def getDriven() -> float: # Methode
            return self.__km_driven
            
        def drive(hours: float):
            self.__km_driven += self.speed * hours
    \end{lstlisting}
    \caption{Klasse Auto in Python}
\end{minipage}
\begin{minipage}{.3\textwidth}
    \begin{tikzpicture}
        \umlclass{Auto}{ 
        + name : Zeichenkette \\ 
        + speed : Dezimalzahl \\ 
        - km\_driven : Dezimalzahl
        }{ 
            + getDriven() : Dezimalzahl \\
            + drive(hours: Dezimalzahl) :
            }
    \end{tikzpicture}
    \caption{UML Klassendiagramm}
\end{minipage}
\end{center}
\end{figure}

\subsection{Vererbung}

Bei der Vererbung wird eine Klasse auf Basis einer anderen Definiert.
Diese Unterklasse übernimmt dann alle Attribute und Methoden der Oberklasse
und kann noch weitere hinzufügen oder neu definieren. Vererbung lässt sich auch
Verketten. Klasse A erbt von B und Klasse B erbt von C. So hat A auch die Attribute
und Methoden von C.

\pgfsetlayers{package0, main, connections}

\vspace*{0.5cm}
\begin{figure}[H]
\begin{center}
    \begin{tikzpicture}
        \umlclass[x=-4,]{Auto}{ 
            + name : Zeichenkette \\ 
            + speed : Dezimalzahl \\ 
            - km\_driven : Dezimalzahl
        }{ 
            + getDriven() : Dezimalzahl \\
            + drive(hours: Dezimalzahl) :
        }
        \umlclass[x=4, y=-1]{Mercedes}{ 
            + preis : Ganzahl \\
            + topspeed: Dezimalzahl
        }{
            + speedPerPrice() : Dezimalzahl\\
        }
        \umlinherit{Mercedes}{Auto}
    \end{tikzpicture}
    \caption{Vererbung}
\end{center}
\end{figure}

\begin{figure}
\begin{lstlisting}[language=python]
class Mercedes(Auto): # Vererbung
    def __init__self(self, name: str, speed: float, preis: Ganzahl, topspeed: Dezimalzahl):
        self.name = name      # geerbtes Attribut
        self.speed = speed    # geerbtes Attribut
        self.__km_driven = 0  # geerbtes Attribut
        self.preis = preis
        self.topspeed = topspeed

    def speedPerPrice() -> float: # neue Methode
        return self.topspeed / self.preis
\end{lstlisting}
\caption{Klasse Mercedes in Python}
\end{figure}

\subsection{Klassendiagramme}

Die Unified Modeling Language (UML) kann für sogenannte UML-Diagramme zur abstrakten
Modellierung genutzt werden (von der Programmiersprache unabhängig).
Im oberen Teil einer Klasse werden die Attribute mit Datentyp definiert.
Im unteren Teil werden die Methoden der Klasse mit Rückgabewert definiert.
Ein Plus bedeutet öffentlich, ein Minus bedeutet Privat, ein Hashtag
bedeutet Geschützt und eine Tilde bedeutet nur im Package sichtbar.
Eine Vererbung wird durch einen Pfeil mit unausgefülltem Ende von der erbenden
zur vererbenden Klasse dargestellt. Eine Assoziation ist einfach eine Linie.

\vspace*{0.5cm}
\begin{figure}[H]
\begin{center}
    \begin{tikzpicture}
        \umlclass[x=-2,y=5]{Autohaus}{ 
            + autos : Reihung vom Typ Auto \\ 
        }{ 
            + getAutoCount() : Ganzzahl \\
        }
        \umlclass[x=-4,]{Auto}{ 
            + name : Zeichenkette \\ 
            + speed : Dezimalzahl \\ 
            - km\_driven : Dezimalzahl
        }{ 
            + getDriven() : Dezimalzahl \\
            + drive(hours: Dezimalzahl) :
        }
        \umlclass[x=4, y=-1]{Mercedes}{ 
            + preis : Ganzahl \\
            + topspeed: Dezimalzahl
        }{
            + speedPerPrice() : Dezimalzahl\\
        }
        \umlassoc{Auto}{Autohaus}
        \umlinherit{Mercedes}{Auto}
    \end{tikzpicture}
    \caption{Klassendiagramme}
\end{center}
\end{figure}

