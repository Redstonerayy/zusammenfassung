\section{Algorithmen}

\subsection{Binäre Suche}

Die Binäre Suche ist ein Algorithmus um in sortierten Datenstrukturen (z. B. Reihungen)
in ein Element zu finden. Dabei werden maximal $log_{2}(n) + 1$ Operationen benötigt,
bis das Element gefunden wurde (oder feststeht, das es nicht enthalten ist).
Die Binäre Suche lässt sich auf lexikographisch sortierbare Datentypen anwenden,
z. B. eine Reihung von Ganzzahlen. Durch vergleichen des gesuchten Werts mit dem mittleren
Wert des aktuellen Bereiches kann festgestellt werden, ob sich dieser Wert im linken oder
rechten Teil des sortierten Bereichs befindet. Anschließend wird mit dem mittleren Wert
des neuen Teilbereichs verglichen und schließlich findet man so das gesuchte Element.
Durch jeden Vergleich wird der Suchbereich halbiert und die nötigen Vergleiche sind
logarithmisch zur Anzahl der Elementanzahl.

\vspace*{0.3cm}
Es soll geprüft werden, ob die Zahl 2 in der sortierten Reihung enthalten ist.
Der aktuelle Vergleich ist rot und der aktuelle Bereich blau markiert.
Nach 4 Vergleichen ist klar, dass 2 enthalten ist. 
Es wurden also $log_{2}(11) + 1$ Versuche benötigt.

% matrix magic
\pgfdeclarelayer{myback}
\pgfsetlayers{myback,background,main}

\tikzset{mycolor/.style = {line width=1bp,color=#1}}%
\tikzset{myfillcolor/.style = {draw,fill=#1}}%

\NewDocumentCommand{\highlight}{O{blue!40} m m}{%
\draw[mycolor=#1] (#2.north west)rectangle (#3.south east);
}

\begin{figure}[H]
    \centering
    \adjustbox{scale=1.5}{
        \begin{tikzpicture}[baseline=-\the\dimexpr\fontdimen22\textfont2\relax ]
        \matrix (m) [
            matrix of math nodes,
            left delimiter={[},
            right delimiter={]},
            row sep=10pt,
            r/.style={fill=red!20},
            b/.style={fill=blue!20}
        ] {
            1 & 2 & 3 & 4 & 5 & |[r]| 6 & 7 & 8 & 9 & 10 & 11 \\
            1 & 2 & |[r]| 3 & 4 & 5 & 6 & 7 & 8 & 9 & 10 & 11 \\
            |[r]| 1 & 2 & 3 & 4 & 5 & 6 & 7 & 8 & 9 & 10 & 11 \\
            1 & |[r]| 2 & 3 & 4 & 5 & 6 & 7 & 8 & 9 & 10 & 11 \\
        };
        
        \begin{pgfonlayer}{myback}
        \highlight[blue]{m-1-1}{m-1-11}
        \highlight[blue]{m-2-1}{m-2-5}
        \highlight[blue]{m-3-1}{m-3-2}
        \highlight[blue]{m-4-2}{m-4-2}
        \end{pgfonlayer}
        \end{tikzpicture}
    }
\end{figure}

\subsection{Quicksort}