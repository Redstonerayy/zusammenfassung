\section{Algorithmen}

\subsection{Binäre Suche}

Die Binäre Suche ist ein Algorithmus um in sortierten Datenstrukturen (z. B. Reihungen)
in ein Element zu finden. Dabei werden maximal $log_{2}(n) + 1$ Operationen benötigt,
bis das Element gefunden wurde (oder feststeht, das es nicht enthalten ist).
Die Binäre Suche lässt sich auf lexikographisch sortierbare Datentypen anwenden,
z. B. eine Reihung von Ganzzahlen. Durch vergleichen des gesuchten Werts mit dem mittleren
Wert des aktuellen Bereiches kann festgestellt werden, ob sich dieser Wert im linken oder
rechten Teil des sortierten Bereichs befindet. Anschließend wird mit dem mittleren Wert
des neuen Teilbereichs verglichen und schließlich findet man so das gesuchte Element.
Durch jeden Vergleich wird der Suchbereich halbiert und die nötigen Vergleiche sind
logarithmisch zur Anzahl der Elementanzahl.

\vspace*{0.3cm}
Es soll geprüft werden, ob die Zahl 2 in der sortierten Reihung enthalten ist.
Der aktuelle Vergleich ist rot und der aktuelle Bereich blau markiert.
Nach 4 Vergleichen ist klar, dass 2 enthalten ist. 
Es wurden also $log_{2}(11) + 1$ Versuche benötigt.

% matrix magic
\pgfdeclarelayer{myback}
\pgfsetlayers{myback,background,main}

\tikzset{mycolor/.style = {line width=1bp,color=#1}}%
\tikzset{myfillcolor/.style = {draw,fill=#1}}%

\NewDocumentCommand{\highlight}{O{blue!40} m m}{%
\draw[mycolor=#1] (#2.north west)rectangle (#3.south east);
}

\begin{figure}[H]
    \centering
    \adjustbox{scale=1.5}{
        \begin{tikzpicture}[baseline=-\the\dimexpr\fontdimen22\textfont2\relax ]
        \matrix (m) [
            matrix of math nodes,
            left delimiter={[},
            right delimiter={]},
            row sep=10pt,
            r/.style={fill=red!20},
            b/.style={fill=blue!20}
        ] {
            1 & 2 & 3 & 4 & 5 & |[r]| 6 & 7 & 8 & 9 & 10 & 11 \\
            1 & 2 & |[r]| 3 & 4 & 5 & 6 & 7 & 8 & 9 & 10 & 11 \\
            |[r]| 1 & 2 & 3 & 4 & 5 & 6 & 7 & 8 & 9 & 10 & 11 \\
            1 & |[r]| 2 & 3 & 4 & 5 & 6 & 7 & 8 & 9 & 10 & 11 \\
        };
        
        \begin{pgfonlayer}{myback}
        \highlight[blue]{m-1-1}{m-1-11}
        \highlight[blue]{m-2-1}{m-2-5}
        \highlight[blue]{m-3-1}{m-3-2}
        \highlight[blue]{m-4-2}{m-4-2}
        \end{pgfonlayer}
        \end{tikzpicture}
    }
\end{figure}

\clearpage

\subsection{Quicksort}

Quicksort ist ein von Tony Hoare publizierter, rekursiver Sortieralgorithmus mit einer
Komplexität von $O(n \cdot log_{2}(n))$, von dem viele unterschiedlich gute Variationen existieren.
Die Grundstruktur aller Versionen ist dabei ähnlich.

\begin{enumerate}
    \item Wenn die Länge des Sortierbereichs kleiner als 2 ist, ist der Bereich sortiert.
    Es kann auch bei anderen Minimalgrenzen (e.g. 10) dann der Bereich mit einem anderen
    Algorithmus sortiert werden.
    \item Ist dies nicht der Fall, wird ein Pivot Element ausgewählt. Dabei gibt es Verschiedene
    Wege der Auswahl (z. B. kann man das Mittlere nehmen).
    \item Anschließend wird der Sortierbereich in 2 Unterbereich aufgeteilt, sodass in einem
    Bereich alle Elemente kleiner oder gleich und im anderen Bereich alle Elemente größer oder gleich
    dem Pivot Element sind.
    \item Anschließend wendet man Quicksort (Schritte 1 bis 4) auf jeden dieser Unterbereich an.
\end{enumerate}

Ein auf Hoares publikation beruhender Pseudocode könnte so aussehen.
Die Funktion $quicksort$ sorgt für das Anwenden der Partitionsmethode und das rekursive
Aufrufen. Hoares Partitionsfunktion nutzt in $partition$ 2 Zeiger, einer startet links,
der andere rechts. Der Linke bewegt sich nach rechts, solange alle Elemente kleiner
als das Pivot Element sind. Der Rechte bewegt sich nach links, solange alle Elemente
größer als das Pivot Elemente sind. Wenn beide Zeiger stoppen und sie sich nicht überkreuzt
haben, so sind jeweils die Elemente an der Zeigerposition im falschen Teil des Bereichs und
müssen getauscht werden. Durch Wiederholung wird so der Bereich in die für Quicksort gewollten
2 Bereich geteilt. Die Trennung der Bereiche liegt dort, wo sich die Zeiger begegnen.

\begin{figure}[H]
\begin{lstlisting}
algorithm quicksort(A, lo, hi) is 
    if lo >= 0 && hi >= 0 && lo < hi then
        b := partition(A, lo, hi) // b is r at the end of partition()
        quicksort(A, lo, b)
        quicksort(A, b + 1, hi) 

algorithm partition(A, lo, hi) is 
    pivot := A[ floor((hi - lo)/2) + lo ] // middle of array

    l := lo - 1
    r := hi + 1

    loop forever 
        do l := l + 1 while A[l] < pivot
        do r := r - 1 while A[r] > pivot

        if l >= r then return r

        swap A[l] with A[r]
\end{lstlisting}
\caption{Pseudocode für Quicksort nach Hoare}
\end{figure}

\begin{figure}[H]
\begin{lstlisting}[language=python]
def quicksort(array, lo, hi):
    if lo >= 0 and hi >= 0 and lo < hi:
        b = partition(array, lo, hi)
        quicksort(array, lo, b)
        quicksort(array, b + 1, hi)


def partition(array, lo, hi):
    pivot = array[int((hi - lo) / 2) + lo]

    l = lo - 1
    r = hi + 1

    while True:
        l += 1
        while arrray[l] < pivot:
            l += 1
        
        r += 1
        while arrray[r] < pivot:
            r += 1

        if l >= r:
            return r

        temp = array[l]
        array[l] = array[r]
        array[r] = temp
\end{lstlisting}
\caption{Python Code für Quicksort nach Hoare}
\end{figure}

\begin{figure}[H]
    % \centering
    \adjustbox{scale=1.2}{
        \begin{tikzpicture}[baseline=-\the\dimexpr\fontdimen22\textfont2\relax]
        \matrix (m) at (0,0) [
            matrix of math nodes,
            left delimiter={[},
            right delimiter={]},
            row sep=10pt,
            r/.style={fill=red!20},
            b/.style={fill=blue!20},
            g/.style={fill=green!20},
            v/.style={fill=violet!20},
            yl/.style={fill=yellow!20},
            row 11 column 4/.style={ nodes = { top color=violet!50, bottom color=green!30 }},
            row 17 column 2/.style={ nodes = { top color=violet!50, bottom color=green!30 }},
            row 19 column 5/.style={ nodes = { top color=violet!50, bottom color=green!30 }}
        ] {
           1 & 8 & 10 & 6 & 7 & 5 & 9 & [1em] |[r]| 6 \\
           |[v]| 1 & 8 & 10 & 6 & 7 & 5 & 9 & [1em] |[r]| 6 \\
           1 & |[v]| 8 & 10 & 6 & 7 & 5 & 9 & [1em] |[r]| 6 \\
           1 & |[v]| 8 & 10 & 6 & 7 & 5 & |[g]| 9 & [1em] |[r]| 6 \\
           1 & |[v]| 8 & 10 & 6 & 7 & |[g]| 5 & 9 & [1em] |[r]| 6 \\
           1 & |[yl]| 5 & 10 & 6 & 7 & |[yl]| 8 & 9 & [1em] |[r]| 6 \\
           1 & 5 & |[v]| 10 & 6 & 7 & |[g]| 8 & 9 & [1em] |[r]| 6 \\
           1 & 5 & |[v]| 10 & 6 & |[g]| 7 & 8 & 9 & [1em] |[r]| 6 \\
           1 & 5 & |[v]| 10 & |[g]| 6 & 7 & 8 & 9 & [1em] |[r]| 6 \\
           1 & 5 & |[yl]| 6 & |[yl]| 10 & 7 & 8 & 9 & [1em] |[r]| 6 \\
           1 & 5 & 6 & 10 & 7 & 8 & 9 & [1em] |[r]| 6 \\
           1 & 5 & |[g]| 6 & |[v]| 10 & 7 & 8 & 9 & [1em] |[r]| 6 \\
           1 & 5 & 6 & 10 & 7 & 8 & 9 & [1em] |[r]| 5 & [0.5em] |[r]| 7 \\
           |[v]| 1 & 5 & 6 & |[v]| 10 & 7 & 8 & 9 & [1em] |[r]| 5 & [0.5em] |[r]| 7 \\
           1 & |[v]| 5 & 6 & |[v]| 10 & 7 & 8 & |[g]| 9 & [1em] |[r]| 5 & [0.5em] |[r]| 7 \\
           1 & |[v]| 5 & |[g]| 6 & |[v]| 10 & 7 & |[g]| 8 & 9 & [1em] |[r]| 5 & [0.5em] |[r]| 7 \\
           1 & 5 & 6 & |[v]| 10 & |[g]| 7 & 8 & 9 & [1em] |[r]| 5 & [0.5em] |[r]| 7 \\
           1 & 5 & 6 & |[yl]| 7 & |[yl]| 10 & 8 & 9 & [1em] & [0.5em] |[r]| 7 \\
           1 & 5 & 6 & 7 & 10 & 8 & 9 & [1em] & [0.5em] |[r]| 7 \\
           1 & 5 & 6 & |[g]| 7 & |[v]| 10 & 8 & 9 & [1em] & [0.5em] |[r]| 7 \\
           1 & 5 & 6 & 7 & 10 & 8 & 9 & [1em] |[r]| 8 & [0.5em] \\
           \dots \\
           1 & 5 & 6 & 7 & 8 & 9 & 10 & [1em] & [0.5em] \\
        };
        \draw[thick,red] ([xshift=0.5em]m-1-7.north east) -- ([xshift=0.5em]m-23-7.south east);
        
        \begin{pgfonlayer}{myback}
            \highlight[blue]{m-1-1}{m-1-7}
            \highlight[blue]{m-2-1}{m-2-7}
            \highlight[blue]{m-3-1}{m-3-7}
            \highlight[blue]{m-4-1}{m-4-7}
            \highlight[blue]{m-5-1}{m-5-7}
            \highlight[blue]{m-6-1}{m-6-7}
            \highlight[blue]{m-7-1}{m-7-7}
            \highlight[blue]{m-8-1}{m-8-7}
            \highlight[blue]{m-9-1}{m-9-7}
            \highlight[blue]{m-10-1}{m-10-7}
            \highlight[blue]{m-11-1}{m-11-7}
            \highlight[blue]{m-12-1}{m-12-7}
            \highlight[blue]{m-13-1}{m-13-3}\highlight[blue]{m-13-4}{m-13-7}
            \highlight[blue]{m-14-1}{m-14-3}\highlight[blue]{m-14-4}{m-14-7}
            \highlight[blue]{m-15-1}{m-15-3}\highlight[blue]{m-15-4}{m-15-7}
            \highlight[blue]{m-16-1}{m-16-3}\highlight[blue]{m-16-4}{m-16-7}
            \highlight[blue]{m-17-1}{m-17-3}\highlight[blue]{m-17-4}{m-17-7}
            \highlight[green]{m-18-1}{m-18-1}\highlight[green]{m-18-2}{m-18-3}\highlight[blue]{m-18-4}{m-18-7}
            \highlight[green]{m-19-1}{m-19-1}\highlight[green]{m-19-2}{m-19-3}\highlight[blue]{m-19-4}{m-19-7}
            \highlight[green]{m-20-1}{m-20-1}\highlight[green]{m-20-2}{m-20-3}\highlight[blue]{m-20-4}{m-20-7}
            \highlight[green]{m-21-1}{m-21-1}\highlight[green]{m-21-2}{m-21-3}\highlight[green]{m-21-4}{m-21-4}\highlight[blue]{m-21-5}{m-21-7}
            \highlight[green]{m-23-1}{m-23-1}\highlight[green]{m-23-2}{m-23-3}\highlight[green]{m-23-4}{m-23-4}\highlight[green]{m-23-5}{m-23-5}\highlight[green]{m-23-6}{m-23-7}
        \end{pgfonlayer}
        
        \matrix (m) at (7, 0) [
            matrix of math nodes,
            left delimiter={[},
            right delimiter={]},
            row sep=10pt,
            r/.style={fill=red!20},
            g/.style={fill=green!20},
            v/.style={fill=violet!20},
            yl/.style={fill=yellow!20},
        ] {
            |[r]| 6 & [0.5em] PivotElement \\
            |[v]| 1 & LeftPointer \\
            |[g]| 9 & RightPointer \\
            |[yl]| 5 & SwapElements \\
            3 & SortRange \\
            7 & FinishedRange \\
        };
        \begin{pgfonlayer}{myback}
            \highlight[blue]{m-5-1}{m-5-1}
            \highlight[green]{m-6-1}{m-6-1}
        \end{pgfonlayer}
        \end{tikzpicture}
    }
    \caption{Beispieldurchlauf des Algorithmus}
\end{figure}

\clearpage

\subsection{Trace-Tabellen}

Eine Trace Tabelle soll die Veränderung der Daten (Variablen)
während der Ausführung eines Programms (Algorithmus) darstellen
und ist vergleichbar mit einem Debugger.

\begin{figure}[H]
\begin{lstlisting}
int x;
for (i = 0; i < 3; i++){
    for(j = 0; j < 2; j++){
        x = i + j; 
        // trace here
    }
}
// trace here
\end{lstlisting}
\end{figure}

\begin{table}[H]
    \begin{tabular}{|l|l|l|}
    \hline
        i & j & x \\ \hline
        0 & 0 & 0 \\ \hline
        0 & 1 & 1 \\ \hline
        1 & 0 & 1 \\ \hline
        1 & 1 & 2 \\ \hline
        2 & 0 & 2 \\ \hline
        2 & 1 & 3 \\ \hline
        ? & ? & 3 \\ \hline
    \end{tabular}
\end{table}
