\documentclass{article}

% various packages and settings {{{

% math packages
\usepackage{amsmath, amsfonts, mathtools}

% other important packages (should stay on top)
\usepackage{graphicx, multicol, xcolor}

% change to German
\usepackage[german]{babel}

% utf8 characters
\usepackage[utf8]{inputenc}

% font
\usepackage[scaled]{helvet}
\renewcommand{\familydefault}{\sfdefault}
\usepackage[]{fontspec}

% fontsize
\usepackage[12pt]{extsizes}

% no paragraph indents
\setlength{\parindent}{0pt}

% better hyphenation
\usepackage[final]{microtype}
\usepackage{csquotes}

% import line spacing
\usepackage{setspace}

% page format
\usepackage{geometry}
\geometry{
    a4paper,
    left=25mm,
    right=25mm,
    top=25mm,
    bottom=25mm
}
    
% page header
\usepackage{fancyhdr}
\pagestyle{fancy}
\fancyhf{}
\lhead{\textbf{\today}}
\rhead{\textbf{Schwingungen}}

% }}}

\begin{document}

% set line spacing
\setstretch{1.5}

\section*{Versuche zu Feder- und Fadenpendel}

Um Aufschluss über die Zusammehänge von Masse (m), Federstärke / Federkonstante (D),
Startamplitude ($s_0$) beim Federpendel und Masse (m), Fadenlänge (l),
Startamplitude ($s_0$) beim Fadenpendel in Bezug auf die zu Erwartende
Schwingungsdauer zu erhalten lassen sich 2 Experimente durchführen.

Faden und Feder lassen sich an einem Stativ befestigen. Anschließend befestigt man ein
Massestück an Faden und Feder, sodass man nun 2 funktionierende Pendel hat.
Zur Messung der Schwingungsdauer (T) läßt sich nun eine Lichtschranke verwenden.
Beim Federpendel bietet es sich dabei an, etwas vom Massestück herausragendes zu nutzen
wie z. B. ein Stück Papier, damit nicht die Feder detektiert wird von der Lichtschranke.

In verschiedenen Experimentteilen lassen sich nun die größen Masse (m),
Federstärke / Federkonstante (D), Fadenlänge (l) und Startamplitude ($s_0$) variieren.
So lassen sich anschließend Messwerte erkennen. Besonders einfach ist dies, wenn man
eine Lichtschranke mit guten digitalen möglichkeiten nutzt, so dass die Daten von einem
digitalen Gerät (e. g. Tablet) direkt z. B. als Graph dargestellt werden.

\clearpage

\section*{Federpendel}

\subsection*{Herleitung der Formel zur Berechnung der Schwingungsdauer (T) beim Federpendel}

Diesem Ansatz liegt der zugrunde, dass beim Federpendel das lineare Kraftgesetz gilt.

\large
$F_R = -D \cdot s_0$
\hspace*{1cm}$F_R \propto s_0$
\vspace{0.5cm}

\normalsize
Die Kraft läßt sich auch durch $F = m \cdot a$ beschrieben.
Die Beschleunigung ist die 2. Ableitunge der Strecke $a = \ddot{s}$. Dies lässt sich
einsetzen.

\large
$m \cdot \ddot{s} = -D \cdot s_0$
\vspace{0.5cm}

\normalsize
Da die Strecke jeweils abhängig zur Zeit ist, lassen sich in diesem Fall $\ddot{s}$ und
$s_0$ als Sinusfunktion mit $t$ als Parameter darstellen.

\large
$m \cdot \ddot{s}(t) = -D \cdot s(t)$
\vspace{0.5cm}

\normalsize
Vereinfacht kann man von einer ungedämpften Schwingung ausgehen, sodass wir eine
normale Sinusfunktion mit $s_0$ als Amplitude haben. Daraus lässt sich dann auch die
Ableitung bilden.

\large
$s(t) = s_0 \cdot \sin (\frac{2\pi}{T} \cdot t)$

$\ddot{s}(t) = s_0 \cdot \sin (\frac{2\pi}{T} \cdot t) \cdot -(\frac{2\pi}{T})^2$
\vspace{0.5cm}

\normalsize
Nun setzt man diese beiden Funktionen ein und formt um.

\large
\begin{align*}
    m \cdot -(\frac{2\pi}{T})^2 \cdot s_0 \cdot \sin (\frac{2\pi}{T} \cdot t) & =
    -D \cdot s_0 \cdot \sin (\frac{2\pi}{T} \cdot t)
                                                                              & | : s_0 \cdot \sin (\frac{2\pi}{T} \cdot t) \\
    m \cdot -(\frac{2\pi}{T})^2                                               & =
    -D                                                                        & | \cdot -1, : m                             \\
    (\frac{2\pi}{T})^2                                                        & =
    \frac{D}{m}                                                               & | \sqrt{\phantom{x}}                        \\
    \frac{2\pi}{T}                                                            & =
    \sqrt{\frac{D}{m}}                                                        & | \text{\normalsize Kehrbrüche}, : 2\pi     \\
    T                                                                         & =
    2\pi \cdot \sqrt{\frac{m}{D}}
\end{align*}

\clearpage

\section*{Fadenpendel}

$T = 2\pi \cdot \sqrt{\frac{l}{g}}$

\end{document}
